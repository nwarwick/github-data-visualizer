\documentclass{article}
% Use this command to override the default ACM copyright statement
% (e.g. for preprints).  Consult the conference website for the
% camera-ready copyright statement.

% Load basic packages
\usepackage{balance}  % to better equalize the last page
\usepackage{graphics} % for EPS, load graphicx instead 
\usepackage{txfonts}
\usepackage{times}    % comment if you want LaTeX's default font
\usepackage[pdftex]{hyperref}
\usepackage{color}
\usepackage{textcomp}
\usepackage{booktabs}
\usepackage{ccicons}
\usepackage{todonotes}
\usepackage{csquotes}
\usepackage{url}
\usepackage{pdfpages}
\usepackage{times}

% llt: Define a global style for URLs, rather that the default one
\makeatletter
\def\url@leostyle{%
  \@ifundefined{selectfont}{\def\UrlFont{\sf}}{\def\UrlFont{\small\bf\ttfamily}}}
\makeatother
\urlstyle{leo}

% To make various LaTeX processors do the right thing with page size.
\def\pprw{8.5in}
\def\pprh{11in}
\special{papersize=\pprw,\pprh}
\setlength{\paperwidth}{\pprw}
\setlength{\paperheight}{\pprh}
\setlength{\pdfpagewidth}{\pprw}
\setlength{\pdfpageheight}{\pprh}

% Make sure hyperref comes last of your loaded packages, to give it a
% fighting chance of not being over-written, since its job is to
% redefine many LaTeX commands.
\definecolor{linkColor}{RGB}{6,125,233}
\hypersetup{
  pdftitle={Programming languages of cities},
  pdfauthor={LaTeX},
  pdfkeywords={},
  bookmarksnumbered,
  pdfstartview={FitH},
  colorlinks,
  citecolor=black,
  filecolor=black,
  linkcolor=black,
  urlcolor=linkColor,
  breaklinks=true,
}

\pagenumbering{arabic}  % Arabic page numbers for submission.  Remove this line to eliminate 

\setcounter{secnumdepth}{3} % Automatically adds section numbers

% Beginning of document.
\begin{document}

\title{Assigning programming languages to cities based on GitHub users activities}

\author{
   Gelera, Rodney \\University of Victoria
   \and
   McCulloch, Kaileen \\University of Victoria
   \and
   Warwick, Nick \\University of Victoria
}

\maketitle

\tableofcontents
\newpage

\begin{abstract}
Data has never before been generated at such high speeds. With the ever increasing volume of data it has become more and more challenging to sift through and extract useful information. Data mining and data visualization are two tools that can help solve this problem. In this paper we are going to use classification to simplify the information and a visualizer for quick and easy evaluation of information.
\end{abstract}

\section{Introduction}
\section{Related Work}
\section{Data Collections}
   \subsection{Scraping}
   \subsection{Mapping}
\section{Data Mining}
\section{Data Visualization}
\section{Conclusion}

% REFERENCES FORMAT
% References must be the same font size as other body text.
\bibliographystyle{SIGCHI-Reference-Format}
\bibliography{../../latex_files/refs}
\citation
\pagebreak
\onecolumn
\begingroup

%\includepdf[scale=.85,pages={1-3},pagecommand=\subsection{Appendix B}]{./figures/questionnaire3.pdf}

\section{Appendix A}
\subsection{Code}
\endgroup

\end{document}

